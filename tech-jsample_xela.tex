%%
%%  提出時はこちらをコメントアウトし,platexでコンパイルする.
%% 
%\documentclass[submit,techreq,noauthor]{ipsj}

%%
%%  ShareLatex など XeLaTex ベースでの執筆時は,こちら(2行)をコメントアウトする.
%%
\documentclass[a4paper]{article}
\usepackage{ipsj}

%以下,使用するパッケージ,マクロなどを書いていく
\usepackage{graphicx} % オプションは不要である.
\usepackage{latexsym}
%\usepackage{amsmath} % 使用を推奨する.
%\usepackage{amssymb} % 使用を推奨する.

\def\Underline{\setbox0\hbox\bgroup\let\\\endUnderline}
\def\endUnderline{\vphantom{y}\egroup\smash{\underline{\box0}}\\}
\def\|{\verb|}

\setcounter{巻数}{53}%vol53=2012, ..., 56=2015, 57=2016, 58=2017...
\setcounter{page}{1}

% マクロを書いていくと便利である.
\newcommand{\fref}[1]{図~\ref{#1}}
\newcommand{\depth}{d} % ノーテーションを変える際などに威力を発揮する.

\begin{document}


\title{ShareLatex を用いた情報処理学会研究報告の共同執筆}


\affiliate{IPSJ}{情報処理学会\\
IPSJ, Chiyoda, Tokyo 101--0062, Japan}


\paffiliate{JU}{情報処理大学\\
Johoshori Uniersity}

\author{情報 太郎}{Joho Taro}{IPSJ}[joho.taro@ipsj.or.jp]
\author{処理 花子}{Shori Hanako}{IPSJ}
\author{学会 次郎}{Gakkai Jiro}{IPSJ,JU}[gakkai.jiro@ipsj.or.jp]

\begin{abstract}
本稿は,情報処理学会の研究会用原稿(情処研報)の執筆をXeLaTexで行えるようにした疑似的なスタイルファイルの説明をまとめたものである.
主に,ShareLatex での同時共同執筆を目的としたものである.
また,論文執筆の注意点についても,情処提供のサンプルファイルから抜粋してある.
あくまで,疑似的なものであるため,提出時は学会提供クラスファイルを使用しなければならないことに注意が必要である.
\end{abstract}


%\begin{jkeyword}
%情報処理学会論文誌ジャーナル,\LaTeX,スタイルファイル,べからず集
%\end{jkeyword}
%
%\begin{eabstract}
%This document is a guide to prepare a draft for submitting to IPSJ
%Journal, and the final camera-ready manuscript of a paper to appear in
%IPSJ Journal, using {\LaTeX} and special style files.  Since this
%document itself is produced with the style files, it will help you to
%refer its source file which is distributed with the style files.
%\end{eabstract}
%
%\begin{ekeyword}
%IPSJ Journal, \LaTeX, style files, ``Dos and Dont's'' list
%\end{ekeyword}

\maketitle

%1
\section{はじめに}

本稿では,ShareLatex上での情報処理学会研究会原稿の共同執筆のためのスタイルファイルの説明を行う.
なお,本プロジェクトは,情報処理学会とは無関係であるので,本スタイルファイルを使用する場合は自己責任であることに注意していただきたい.
また,スタイルの正確性は保証していないので,提出時は,情処提供クラスファイルを用いてローカルでコンパイルすること.

本スタイルファイルは,情処研報2012年10月12日版を参考に作成してある.
研究会用原稿\\\verb|\documentclass[submit,techreq,noauthor]{ipsj}|\\にのみ対応している.

\subsection{使い方}
基本的には,情処提供のサンプル通りに執筆していけばよい.
ただし,
\\\verb|\documentclass[submit,techreq,noauthor]{ipsj}|\\
をコメントアウトし,代わりに以下の2行を記述する.
\\\verb|\documentclass[a4paper]{article}|\\
\verb|\usepackage{ipsj}|\\
以下のコマンドを実行すると,PDF原稿が生成される.
\\\verb|xelatex tech-jsample_xela.tex|

提出時は,上記変更を元に戻し,情処提供クラスファイルを用いて\verb|platex|でコンパイルすること.
文字コードに注意すること.Windowsの場合は,Shift-JISに変換する必要がある.
\verb|compile_original.bat| も参考にすること.

\subsection{ShareLatex の設定}
ShareLatex には以下4つのファイルをアップロードし,左上のメニューから,XeLatexをコンパイラに指定すれば,ShareLatex上での同時共同執筆が可能である.
\begin{itemize}
\item \verb|ipag.ttf|
\item \verb|ipamjm.ttf|
\item \verb|ipsj.sty|
\item \verb|tech_jsample_xela.tex|
\end{itemize}
ShareLatex 上では日本語ファイルも使用可能であるが,ローカルでのコンパイルでうまくいかないことがあるので,ファイル名は半角英数のみを使用すること.

\subsection{既知の問題}
\begin{itemize}
\item 著者名の間のスペースが詰まる.
\item 小さい文字(ャなど)で文字間隔が詰まる.
\item subsection, subsubsection の後の行間が開く.
\item itemize 環境で行間が開く.
\item biography の無視はできない(エラーとなる).
\item ヘッダーに線が入っている.
\item その他細かいレイアウトのずれが発生する.
\end{itemize}

\subsection{免責}
本スタイルファイルを用いたことによるいかなる損害についてもその責を負いません.
自己責任にてご使用ください.また,スタイルの正確性は保証していません.
提出時は必ず学会提供クラスファイルを用いてコンパイルしてください.
\clearpage
%2
\section{投稿の流れ}
以降は,学会提供のサンプルの内容(一部変更有)である.

%2.1
\subsection{準備}

情報処理学会論文誌ジャーナルの \LaTeX スタイルファイルを含む論文執筆キッ
トは
\begin{quote}
\small
\|http://www.ipsj.or.jp/jip/submit/style.html|
\end{quote}
からダウンロードすることができる.
キットはUnix用,Windows (DOS)用,Macintosh用などが用意されており,著者の
作業環境に応じたものを選択できるようになっている.また,実行環境としては 
\LaTeXe を前提としているので,準備されたい.

本XeLaTeX疑似スタイルには,上記キットのうちの,和文研究報告用スタイルファ
イルが同梱してある.

%2.2
\subsection{最終原稿の作成と投稿}

本稿に従って用意した投稿用原稿の \LaTeX ソースからpdfファイルを作成し,
Adobeのpdf readerで読めることを確認した後,
\begin{quote}
\small
\|https://www.ipsj.or.jp/prms/author_pre_submit.do|
\end{quote}
のPRMS (Paper Review Management System)にメールアドレスを登録し,送られ
たきたメールに従って,指定されたURLから投稿する.投稿の流れについては,
\begin{quote}
\small
\|http://www.ipsj.or.jp/journal/submit/manual/|
\|manual_j_for_Author.pdf|
\end{quote}
を参照されたい.


%2.3
\subsection{最終原稿の作成とファイルの送付}
図表など
のレイアウトを最終的なものとする.なお後の校正の手間を最小にするために,
この段階で記述の誤りなどを完全に除去するように綿密にチェックして頂きたい.

最終版では,著者名およびその所属を表示すると同時に,学会より指示された巻
数,号数,先頭ページ番号,受付/採録年月日(年は西暦)を記述する.なお学
会からの指示がない項目に関しては,記述しなくてよい.

ファイルの送付方法などについては,採録通知とともに学会事務局から送られる
指示に従う.

%2.4
\subsection{著者校正・組版・出版}

学会では用語や用字を一定の基準に従って修正することがある.また \LaTeX の
実行環境の差異などによって著者が作成したハードコピーと実際の組版結果が微
妙に異なることがある.これらの修正や差異が問題ないかを最終的に確認するた
めに,著者にゲラ刷りが送られるので,もし問題があれば朱書によって指摘して
返送する.なお{\bf この段階での記述誤りの修正は原則として認められない}の
で,原稿送付時に細心の注意を払っていただきたい.

その後,著者の校正に基づき最終的な組版を行ない,オンライン出版する.


%3
\section{論文フォーマットの指針}
\label{sec:format}

以下,情報処理学会論文誌ジャーナル用スタイルファイルを用いた論文フォーマッ
トの指針について述べるので,これに従って原稿を用意頂きたい.\LaTeX を用
いた一般的な文章作成技術については,\cite{okumura, companion} 等を参考に
されたい.

%4

%4.2
\subsection{表題・著者名等}

表題,著者名とその所属,および概要を前述のコマンドや環境により{\bf 和文と
英文の双方について}定義した後,\|\maketitle| によって出力する.

%4.2.1
\subsubsection{表題} 

表題は,\|\title| および \|\etitle| で定義した表題はセンタリングされる.
文字数の多いものについては,適宜 \|\\| を挿入して改行する.

%4.2.2
\subsubsection{著者名・所属} 

各著者の所属を第一著者から順に \|\affiliate| を用いてラベル(第1引数)を
付けながら定義すると,脚注に番号を付けて所属が出力される.なお,複数の著
者が同じ所属である場合には,一度定義するだけで良い.

現在の所属は \|\paffiliate| を用い,同様にラベル,所属先を記述する.所属
先には自動で「現在」,\|\\|の改行で「Presently with」が挿入される.著者
名は \|\author| で定義する.各著者名の直後に,英文著者名,所属ラベルとメー
ルアドレスを記入する.著者が複数の場合は \|\author| を繰り返すことで,2
人,3人,\dots と増えていく.現在の所属や,複数の所属先を追加する場合に
は,所属ラベルをカンマで区切り,追加すればよい.

また,メールアドレス部分は省略が可能だが,必ず代表者のアドレスは必要とな
る.なお,和文著者名,英文著者名は,姓と名を半角(ASCII)の空白で区切る.

%4.2.3
\subsubsection{概要} 

和文の概要は \|abstract| 環境の中に記述する.

%4.3
\subsection{本文}

%4.3.1
\subsubsection{見出し}

節や小節の見出しには \|\section|, \|\subsection|, \|\subsubsection|,
\|\paragraph| といったコマンドを使用する.

「定義」,「定理」などについては,\|\newtheorem|で適宜環境を宣言し,そ
の環境を用いて記述する.

%4.3.2
\subsubsection{行送り}

2段組を採用しており,左右の段で行の基準線の位置が一致することを原則とし
ている.また,節見出しなど,行の間隔を他よりたくさんとった方が読みやすい
場所では,この原則を守るようにスタイルファイルが自動的にスペースを挿入す
る.したがって本文中では \|\vspace| や \|\vskip| を用いたスペースの調整
を行なわないようにすること.

%4.3.3
\subsubsection{フォントサイズ}

フォントサイズは,スタイルファイルによって自動的に設定されるため,基本的
には著者が自分でフォントサイズを変更する必要はない.

%4.3.4
\subsubsection{句読点}

句点には全角の「.」,読点には全角の「,」を用いる.ただし英文中や数式中
で「.」や「,」を使う場合には,半角文字を使う.「。」や「、」は使わない.

%4.3.5
\subsubsection{全角文字と半角文字}

全角文字と半角文字の両方にある文字は次のように使い分ける.

\begin{enumerate}
\item 括弧は全角の「(」と「)」を用いる.但し,英文の概要,図表見出し,
書誌データでは半角の「(」と「)」を用いる.

\item 英数字,空白,記号類は半角文字を用いる.ただし,句読点に関しては,
前項で述べたような例外がある.

\item カタカナは全角文字を用いる.

\item 引用符では開きと閉じを区別する.
開きには \|``| を用い,閉じには\|''| を用いる.
\end{enumerate}

%4.3.6
\subsubsection{箇条書}

箇条書に関する形式を特に定めていない.場合に応じて標準的な \|enumerate|,
\|itemize|, \|description| の環境を用いてよい.


%4.3.7
\subsubsection{脚注}

脚注は \|\footnote| コマンドを使って書くと,ページ単位に\footnote{脚注の
例.}や\footnote{二つめの脚注.}のような参照記号とともに脚注が生成される.
なお,ページ内に複数の脚注がある場合,参照記号は \LaTeX を2回実行しない
と正しくならないことに注意されたい.

また場合によっては,脚注をつけた位置と脚注本体とを別の段に置く方がよいこ
ともある.この場合には,\|\footnotemark| コマンドや \|\footnotetext| コ
マンドを使って対処していただきたい.

なお,脚注番号は論文内で通し番号で出力される.

%4.3.8
\subsubsection{OverfullとUnderfull}

組版時にはoverfullを起こさないことを原則としている.従って,まず提出する
ソースが著者の環境でoverfullを起こさないように,文章を工夫するなどの最善
の努力を払っていただきたい.但し,\|flushleft| 環境,\|\\|,
\|\linebreak| などによる両端揃えをしない形でのoverfullの回避は,できるだ
け避けていただきたい.また著者の執筆時点では発生しないoverfullが,組版時
の環境では発生することもある.このような事態をできるだけ回避するために,
文中の長い数式や \|\verb| を避ける,パラグラフの先頭付近では長い英単語を
使用しない,などの注意を払うようにして頂きたい.

%4.4
\subsection{数式}\label{sec:Item}

%4.4.1
\subsubsection{本文中の数式}

本文中の数式は \|$| と \|$|, \|\(| と \|\)|, あるいは \|math| 環境のいず
れで囲んでもよい.

%4.4.2
\subsubsection{別組の数式}

別組数式(displayed math)については \|$$| と \|$$| は使用せずに,\|\[| と 
\|\]| で囲むか,\|displaymath|, \|equation|, \|eqnarray| のいずれかの環
境を用いる.これらは
%
\begin{equation}
\Delta_l = \sum_{i=l|1}^L\delta_{pi}
\end{equation}
%
のように,センタリングではなく固定字下げで数式を出力し,かつ背が高い数式
による行送りの乱れを吸収する機能がある.

%4.4.3
\subsubsection{eqnarray環境}

互いに関連する別組の数式が2行以上連続して現れる場合には,単に\|\[| と 
\|\]|,あるいは \|\begin{equation}| と\|\end{equation}| で囲った数式を書
き並べるのではなく,\|\begin|\allowbreak\|{eqnarray}| と 
\|\end{eqnarray}| を使って,等号(あるいは不等号)の位置で縦揃えを行なっ
た方が読みやすい.

%4.4.4
\subsubsection{数式のフォント}

\LaTeX が標準的にサポートしているもの以外の特殊な数式用フォントは,でき
るだけ使わないようにされたい.どうしても使用しなければならない場合には,
その旨申し出て頂くとともに,組版工程に深く関与して頂くこともあることに留
意されたい.

\begin{figure}[tb]
\setbox0\vbox{
\hbox{\|\begin{figure}[tb]|}
\hbox{\quad \|<|図本体の指定\|>|}
\hbox{\|\caption{<|和文見出し\|>}|}
\hbox{\|\label{| $\ldots$ \|}|}
\hbox{\|\end{figure}|}
}
\centerline{\fbox{\box0}}
\caption{1段幅の図}
\label{fig:single}
\end{figure}

%4.5
\subsection{図}

1段の幅におさまる図は,図\ref{fig:single} の形式で指定する.位置の指定
に \|h| は使わない.また,図の下に和文と英文の双方の見出しを,
\|\caption| と \|\ecaption| で指定する.文字数が多い見出しはは自動的に改
行して最大幅の行を基準にセンタリングするが,見出しが2行になる場合には適
宜 \|\\| を挿入して改行したほうが良い結果となることがしばしばある
(図\ref{fig:single} の英文見出しを参照).図の参照は \|図\ref{<|ラベ
ル\|>}| を用いて行なう.

\begin{figure*}[tb]
\setbox0\vbox{\large
\hbox{\|\begin{figure*}[t]|}
\hbox{\quad \|<|図本体の指定\|>|}
\hbox{\|\caption{<|和文見出し\|>}|}
\hbox{\|\label{| $\ldots$ \|}|}
\hbox{\|\end{figure*}|}}
\centerline{\fbox{\hbox to.9\textwidth{\hss\box0\hss}}}
\caption{2段幅の図}
\label{fig:double}
\end{figure*}


また紙面スペースの節約のために,1つの \|figure|(または \|table|)環境の
中に複数の図表を並べて表示したい場合には,図\ref{fig:left} と 
表\ref{tab:right} のように個々の図表と各々の \|\caption|/\|\ecaption| 
を \|minipage| 環境に入れることで実現できる.なお図と表が混在する場合,
\|minipage| 環境の中で\|\CaptionType{figure}| あるいは \|\CaptionType|
\|{table}| を指定すれば,外側の環境が \|figure| であっても \|table| であっ
ても指定された見出しが得られる.

2段の幅にまたがる図は,図\ref{fig:double} の形式で指定する.
位置の指定は \|t| しか使えない.

図の中身では本文と違い,どのような大きさのフォントを使用しても構わない
(図\ref{fig:double} 参照).また図の中身として,encapsulate された
PostScriptファイル(いわゆるEPSファイル)を読み込むこともできる.読み込
みのためには,プリアンブルで
%
\begin{quote}
\|\usepackage{graphicx}|
\end{quote}
%
を行った上で,\|\includegraphics| コマンドを図を埋め込む箇所に置き,その
引数にファイル名(など)を指定する.

%4.6
\subsection{表}

表の罫線はなるべく少なくするのが,仕上がりをすっきりさせるコツである.罫
線をつける場合には,一番上の罫線には二重線を使い,左右の端には縦の罫線を
つけない (表\ref{tab:example}).表中のフォントサイズのデフォルトは
\|\footnotesize|である.

また,表の上に和文と英文の双方の見出しを, \|\caption|と \|\ecaption| で
指定する.表の参照は \|表\ref{<|ラベル\|>}| を用いて行なう.

\begin{table}[tb] 
\caption{表の例} 
\label{tab:example}
\hbox to\hsize{\hfil
\begin{tabular}{l|lll}\hline\hline
& column1 & column2 & column3 \\\hline
row1 &	item 1,1 & item 2,1 & ---\\
row2 &	---      & item 2,2 & item 3,2 \\
row3 &	item 1,3 & item 2,3 & item 3,3 \\
row4 &	item 1,4 & item 2,4 & item 3,4 \\\hline
\end{tabular}\hfil}
\end{table}


\newpage%%%!!!

%4.7
\subsection{参考文献・謝辞}

%4.7.1
\subsubsection{参考文献の参照}

本文中で参考文献を参照する場合には%,%参考文献番号が文中の単語として使われ
%る場合と,そうでない参照とでは,使用する文字の大きさが異なる.前者は
%\|\Cite|により参照し,後者は
%\|\cite|により参照する.たとえば;
\|\cite|を使用する.参照されたラベルは自動的にソートされ,
\|[]|でそれぞれ区切られる.
%
\begin{quote}
文献 \|\cite{companion,okumura}| は \LaTeX の総合的な解説書である.
\end{quote}
%
と書くと;
%
\begin{quote}
文献\cite{companion,okumura}は \LaTeX の総合的な解説書である.
\end{quote}
%
が得られる.

%4.7.2
\subsubsection{参考文献リスト}
参考文献リストには,
原則として本文中で引用した文献のみを列挙する.
順序は参照順あるいは第一著者の苗字のアルファベット順とする.
文献リストはBiB\TeX と\verb+ipsjunsrt.bst+(参照順)
または\verb+ipsjsort.bst+(アルファベット順)を用いて作り,
\verb+\bibliograhpystyle+と\verb+\bibliography+コマンドにより
利用することが出来る.
これらを用いれば,
規定の体裁にあったものができるので,
できるだけ利用していただきたい.
また製版用のファイル群には\verb+.bib+ファイルではなく\verb+.bbl+ファイルを
必ず含めることに注意されたい.
一方,何らかの理由でthebibliography環境で文献リストを
「手作り」しなければならない場合は,
このガイドの参考文献リストを注意深く見て,
そのスタイルにしたがっていただきたい.




%4.7.3
\subsubsection{謝辞}

謝辞がある場合には,参考文献リストの直前に置き,\|acknowledgment|環境の
中に入れる.この環境の中身は投稿時には出力されない.


%5
\section{論文内容に関する指針}

論文の内容について,論文誌ジャーナル編集委員会で作成した「べからず集」を
以下に示す.投稿前のチェックリストとして利用頂きたい.これ以外にも,査読
者用,メタ査読者用の「べからず集」\cite{webpage2}も公開しているので,参
照されたい.また,作文技術に関する \cite{book1, book2, book3, book4}のよ
うな書籍も参考になる.

%5.1
\subsection{書き方の基本}

\begin{itemize}
 \item  研究の新規性,有用性,信頼性が読者に伝わるように記述する.
 \item  読み手に,読みやすい文章を心がける(内容が前後する,背景・
	       課題の設定が不明瞭などは読者にとって負担).
 \item  解決すべき問題が汎用化(一般的に記述)されていないのは再
	       考を要する(XX大学の問題という記述に終始).あるいは,
	       (単に「作りました」だけで)解決すべき問題そのものの記述
	       がないのは再考を要する.
 \item  結論が明確に記されていない,または,範囲,限界,問題点な
	       どの指摘が適切ではない,または,結論が内容にそったもので
	       はないものは再考を要する.
 \item  科学技術論文として不適当な表現や,分かりにくい表現がある
	       のは再考を要する.
 \item  極端な口語体や,長文の連続などは再考を要する.
 \item  章,節のたて方,全体の構成等が適切でない文章は再考を要す
	       る.
 \item  文中の文脈から推測しないと内容の把握が困難な論文にしない.
 \item  説明に飛躍した点があり,仮説等の説明が十分ではないのは再
	       考を要する.
 \item  説明に冗長な点,逆に簡単すぎる点があるのは再考を要する.
 \item  未定義語を減らす.
\end{itemize}


%5.2
\subsection{新規性と有効性を明確に示す}

\begin{itemize}
 \item  在来研究との関連,研究の動機,ねらい等が明確に説明されて
	       いないのは再考を要する.
 \item  既知/公知の技術が何であって,何を新しいアイデアとして提
	       案しているのかが書かれていないのは再考を要する.
 \item  十分な参考文献は新規性の主張に欠かせない.
 \item  提案内容の説明が,概念的または抽象的な水準に終始していて,
	       読者が提案内容を理解できない(それだけで新規性が感じられ
	       ないもの)のは再考を要する.
 \item  論文で提案した方法の有効性の主張がない,またはきわめて貧
	       弱なのは再考を要する.
\end{itemize}

%5.3
\subsection{書き方に関する具体的な注意}

\begin{itemize}
 \item  和文標題が内容を適切に表現していないのは再考を要する.
 \item  英文標題が内容を適切に表現していない,または英語として適
	       切でないのは再考を要する.
 \item  アブストラクトが主旨を適切に表現していない,または英文が
	       適切ではないのは再考を要する.
 \item  記号・略号等が周知のものでなく,または,用語が適切でなく,
	       または,図・表の説明が適当ではないのは再考を要する.
 \item  個人的あるいは非常に小さなグループ/企業だけで通用するよ
	       うな用語が特別な説明もなしに多用されているのは再考を要す
	       る.
 \item  図表自体は十分に明確ではない,または誤りがあるのは再考を
	       要する.
 \item  図表が鮮明ではないのは再考を要する.
 \item  図表が大きさ,縮尺の指定が適切でないのは再考を要する.
\end{itemize}

%5.4
\subsection{参考文献}

\begin{itemize}
 \item  参考文献は10件以上必要(分野によっては20件以上,30件以上
	       という意見もある).
 \item  十分な参考文献は新規性の主張に欠かせない.
 \item  適切な文献が引用されておらず,その数も適切ではないのは再
	       考を要する.
 \item  日本人によるしかるべき論文を引用することで日本人研究コミュ
	       ニティの発展につながる.
 \item  参考文献は自分のものばかりではだめ.
\end{itemize}

%5.5
\subsection{二重投稿}

\begin{itemize}
 \item  二重投稿はしてはならない ─ ただし国際会議に採択された論
	       文を著作権が問題にならないように投稿することは構わない.
 \item  他の論文とまったく同じ図表を引用の明示なしに利用すること
	       は禁止.
 \item  既発表の論文等との間に重複があるのは再考を要する.
\end{itemize}

%5.6
\subsection{他の人に読んでもらう}

\begin{itemize}
 \item  投稿経験が少ない人は,採録された経験の豊富な人に校正して
	       もらう.
 \item  読者の立場から見て論理的な飛躍がないかに注意して記述する.
\end{itemize}

%5.7
\subsection{その他}

\begin{itemize}
 \item  条件付採録後の修正で,採録条件以外を理由もなく修正するこ
	       とは禁止.
 \item  ダブルブラインドなので査読者は選べない.
 \item  投稿前にチェックリストの各項目を満たしているか,必ず確認
	       する. 
\end{itemize}

%6
\section{おわりに}

本稿では,A4縦型2段組み用に変更したスタイルファイルを用いた論文のフォー
マット方法と,論文誌ジャーナル編集委員会がまとめた「べからず集」に基づく
論文の書き方を示した.内容的にまだ不十分の部分が多いため,意見,要望等を
\begin{quote}
 \|editt@ipsj.or.jp|
\end{quote}
までお寄せ頂きたい.



\begin{acknowledgment}
A4横型に対するガイドを基に,本稿を作成した.
クラスファイルの作成においては,
京都大学の中島 浩氏にさまざまなご教示を頂き,
さらにBiB\TeX 関連ファイルの利用についても快諾頂いたことを深謝する.
また,A4横型に対するガイドを作成された当時の編集委員会の担当者に深謝する.
\end{acknowledgment}



\begin{thebibliography}{10}

%\bibitem{latex}
%Lamport, L.: {\em A Document Preparation System \LaTeX User's Guide \&
%  Reference Manual}, Addison Wesley, Reading, Massachusetts (1986).
% (Cooke, E., et al.訳:文書処理システム \LaTeX,アスキー出版局
%  (1990)).

%\bibitem{total}
%伊藤和人: \LaTeX トータルガイド,秀和システムトレーディング (1991).
%\bibitem{nodera}
%野寺隆志:楽々 \LaTeX,共立出版 (1990).

\bibitem{okumura}
奥村晴彦:改訂第5版 \LaTeXe 美文書作成入門,
技術評論社(2010).

\bibitem{companion}
Goossens, M., Mittelbach, F. and Samarin, A.:
{\it The LaTeX Companion},
Addison Wesley, Reading, Massachusetts (1993).

\bibitem{book1}
木下是雄:
理科系の作文技術,
中公新書(1981).

\bibitem{book2}
Strunk W. J. and White E.B.:
{\it The Elements of Style, Forth Edition},
Longman (2000).

\bibitem{book3}
Blake G. and Bly R.W.:
{\it The Elements of Technical Writing},
Longman (1993).

\bibitem{book4}
Higham N.J.:
{\it Handbook of Writing for the Mathematical Sciences},
SIAM (1998).

\bibitem{webpage1}
情報処理学会論文誌ジャーナル編集委員会:
投稿者マニュアル(online),
\urlj{http://www.ipsj.or.jp/journal /submit/manual/j\_manual.html}
(2007.04.05).

\bibitem{webpage2}
情報処理学会論文誌ジャーナル編集委員会:
べからず集(online),
\urlj{http://www.ipsj.or.jp/journal/manual /bekarazu.html}
(2011.09.15).

\end{thebibliography}



\pagebreak%%!!!
\vspace*{-\baselineskip}%%!!!

\appendix
%7
\section{付録の書き方}

付録がある場合には,参考文献リストの直後にコマンド \|\appendix| に引き続
いて書く.付録では,\|\section| コマンドが{\bf A.1},{\bf A.2}などの見出
しを生成する.

%7.1
\subsection{見出しの例}

付録の \|\subsetion| ではこのよう見出しになる.

%8

%% 以下は無視されます


\end{document}
